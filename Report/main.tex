%%%%%%%%%%%%%%%%%%%%%%%%%%%%%%%%%%%%%%%%%
% Wenneker Article
% LaTeX Template
% Version 2.0 (28/2/17)
%
% This template was downloaded from:
% http://www.LaTeXTemplates.com
%
% Authors:
% Vel (vel@LaTeXTemplates.com)
% Frits Wenneker
%
% License:
% CC BY-NC-SA 3.0 (http://creativecommons.org/licenses/by-nc-sa/3.0/)
%
%%%%%%%%%%%%%%%%%%%%%%%%%%%%%%%%%%%%%%%%%

%----------------------------------------------------------------------------------------
%	PACKAGES AND OTHER DOCUMENT CONFIGURATIONS
%----------------------------------------------------------------------------------------

\documentclass[12pt, a4paper, twocolumn]{article} % 10pt font size (11 and 12 also possible), A4 paper (letterpaper for US letter) and two column layout (remove for one column)
\usepackage{setspace}
\usepackage{multirow}
\usepackage{float}
\usepackage{hyperref}
\usepackage[USenglish,UKenglish,french,spanish,italian]{babel}
\usepackage{graphicx}
\graphicspath{ {./Immagini/} }

\input{structure.tex} % Specifies the document structure and loads requires packages

%----------------------------------------------------------------------------------------
%	ARTICLE INFORMATION
%----------------------------------------------------------------------------------------

\title{Financial Market Analytics Project} % The article title

\author{
    Ruben Agazzi 844736\\
		Davide Abete 882299
}

%----------------------------------------------------------------------------------------

\begin{document}

\selectlanguage{USenglish}

% \maketitle % Print the title

\thispagestyle{firstpage} % Apply the page style for the first page (no headers and footers)

%----------------------------------------------------------------------------------------
%	ABSTRACT
%----------------------------------------------------------------------------------------
\twocolumn[
  \begin{@twocolumnfalse}
    \maketitle
    \begin{abstract}
The project consist in the creation of portfolios, utilizing FTSE MIB index stocks, the main benchmark index of Italian stock markets, using different criteria in order to select the assets of the portfolios, such as R-Squared, Beta Coefficient, Residual variance, Stock variance and cross sectional momentum.
Finally, last step concerns the analisys of the returns and risks of the portfolios.
			\end{abstract}
  \end{@twocolumnfalse}]

\bigskip

\clearpage
% \lettrineabstract{x}
\bigskip
\tableofcontents

%----------------------------------------------------------------------------------------
%	ARTICLE CONTENTS
%----------------------------------------------------------------------------------------


	
	% Select between one and six entries from the list of approved keywords.
	% Don't make up new ones.
	
	%%%%%%%%%%%%%%%%%%%%%%%%%%%%%%%%%%%%%%%%%%%%%%%%%%
	
	%%%%%%%%%%%%%%%%% BODY OF PAPER %%%%%%%%%%%%%%%%%%
	\tableofcontents
	\section{Introduction} 
	In this project, the goal consists in creating different porfolios, using FTSE MIB index stocks, using different criteria in order to select the assests of the portfolios. Finally we make some analisys on the created portfolios, such as the returns and the level of risk of the portfolios, in order to see which one is the better performing.
	\section{Dataset}
	The dataset used consists in past data about the FTSE MIB index, in particular is composed of daily data about the past 5 years of every singular stock present in the FTSE MIB index, obtained using the yahoo finance API.
	\subsection{Dataset Columns}
	The dataset is composed by the following columns:
	\begin{itemize}
		\item Date: Date relative to the datas of the singular stock.
		\item Open: Opening price of the stock.
		\item High: Highest price reached by the stock in the current day.
		\item Low: Lowest price reached by the stock in the current day.
		\item Close: Closing price of the stock.
		\item Volume: Trading volume of the stocks.
		\item Adjusted Close: Closing price adjusted after accounting for any corporate actions.
		\item log ret: This column is calculated using the adjusted close prices, is the logarithm of the adjusted close price of the current day of the stock subtracted by the logarithm of the adjusted close price of the previous day.
	\end{itemize}
	
	\subsection{Data Exploration}
	The data did not present missing data so all the data is used inside the project.
	
	\section{Parameters}
	Afetr obtaining the data, we proceeded to obtain some parameters relative to the single stocks, in order to create the portfolios.
	\subsection{Rolling Regression}
	In order to obtain some of the parameters needed we proceeded to do a step of rolling regression on every single stock. The rolling regression was made using data about the past 180 days, and repeated for every week.
	The rolling regression is made using the Security Market Line(SML):
	\[
		r_i = \alpha_i +\beta_i(R_M)+e_i
	\]
	
	From the various rolling regressions we obtain the following parameters:
	\begin{itemize}
		\item Beta: is the beta coefficient obtained directly from the regression, this parameter indicates the Systematic Risk, in other words the risk that cannot be diversified away.
		\item Residual Variance($\sigma_{ei}$): Is the variance of the residuals of the regression.
		\item R-Squared: R-Squared statistic obtained from the rolling regression, indicates how well the regression can fit the data.
	\end{itemize}
	We decided to not use also the $\alpha$ coefficient because in most of the regressions it wasn't statistically significative.
	\subsection{Other parameters}
	The other parameters used to build the portfolios are:
	\begin{itemize}
		\item Log returns: weekly logarithmic returns of the single stock.
		\item Risk: weekly risk of the single stock, obtained by calculating the variance of the weekly returns of the stock.
	\end{itemize}
	
	\section{Portfolio Selection}
	In this phase we selected five different portfolios using different criteria. In general the portfolios are selected by ordering the weekly stock parameters in decrescent order, and selecting the top and bottom 10\% of the ordered stocks.
	The parameters used for the portfolio creation are:
	\begin{itemize}
		\item Beta Coefficient
		\item Stock variance
		\item Stock returns
		\item Residual variance
		\item R-Squared
	\end{itemize}
	\section{Portfolios Data}
	Using the mentioned parameters we obtained 5 different portfolios. For every portfolio we have obtained the returns and the risks.
	\subsection{Returns}
	In order to calculate the weekly return of one portfolio for the week i analysis we took the logarithmic return at the starting day of the next week and we substract it by the log return at the start of the week.
	\[
		WR_t = LogRet_{t+1} - LogRet_{t-7}
	\]
	\subsection{Risk}
	In order to calculate the weekly risk of the portfolio we calculate the variance-covariance matrix,
	\subsection{Portfolio returns}
	In order to obtain the portfolio returns we used the calculated weekly logarithmic returns, we made the inverse transformation in order to obtain the weekly return and multiplicated for the available investment.
	\[
	r_{t}	= r_{t-1}*e^{logret_{t}}
	\]
	The portfolio annual return is obtained multiplicating by 52 the weekly portfolios returns
	\subsection{Beta Portfolio}
	This portfolio is obtained by taking the top and bottom 10\% of the stocks, odered by the beta parameter, and is rebalanced every week.\\
	The portfolio expected annual returns are equal to 9.1\%, while the annual risk associated to the portfolio is 36.23\%. The efficiency of the portfolio is equal to 27.99\%.
	\subsection{R-Squared Portfolio}
	This portfolio is obtained by taking the top and bottom 10\% of the stocks, odered by the R-Squared parameter, and is rebalanced every week.\\
	The portfolio expected return,on the five years analized, is 102.40 euros, with an initial investment of 100 euros. The portfolio risk, calculated on the returns of a 100 euros investment, is equal to 4.33.
	\subsection{Stock variance Portfolio}
	This portfolio is obtained by taking the top and bottom 10\% of the stocks, odered by their stock return variance, and is rebalanced every week.\\
	The portfolio expected return,on the five years analized, is 106.62 euros, with an initial investment of 100 euros. The portfolio risk, calculated on the returns of a 100 euros investment, is equal to 4.50.
	\subsection{Residual variance Portfolio}
	This portfolio is obtained by taking the top and bottom 10\% of the stocks, odered by their rolling regression residual variance, and is rebalanced every week.\\
	The portfolio expected return,on the five years analized, is 102.40 euros, with an initial investment of 100 euros. The portfolio risk, calculated on the returns of a 100 euros investment, is equal to 4.67.
	\subsection{Weekly returns Portfolio}
	This portfolio is obtained by taking the top 20\% of the stocks, odered by their weekly returns, and is rebalanced every week.\\
	This aproach is a strategy called ross Sectional Momentum Strategy, which consists in selecting the stocks of the portfolio by looking at past returns ant taking only the best performing stocks, in this case we used the previous week best performing stocks.
	The portfolio expected return,on the five years analized, is 108.99 euros, with an initial investment of 100 euros. The portfolio risk, calculated on the returns of a 100 euros investment, is equal to 4.76.
	\subsection{FTSE-MIB Portfolio}
	This portfolio is obtained by following the FTSE-MIB index.
	The portfolio expected return,on the five years analized, is 99.27 euros, with an initial investment of 100 euros. The portfolio risk, calculated on the returns of a 100 euros investment, is equal to 2.52.
	
	\section{Results}
		Finally we made a table and a chart in order to compare all the portfolios.
	\begin{table}[H]
		\centering
		\caption{Table containing the expected returns and risks of the created portfolios, calculated with an initial investment of 100 euros}
		\begin{tabular}{cccc} % four columns, alignment for each
			\hline
			/ & A. Ret(\%) & A. Risk(\&) & Eff.(\%)\\
			\hline
			Beta 			& 9.1 			& 36.23 & 27.99\\
			R-Squared & 15.2 		& 32.51 & 42.1\\
			Variance 	& 6.08 		& 35.25 & 17.26\\
			Res. Var. & 15.2		& 35.08 & 43.48\\
			Return 		& -114.18 & 34.27 & -417.04\\
			FTSE MIB 	& 0.26 		& 8.63 	& 3.07\\
		
			\hline
		\end{tabular}
	\end{table}

		\begin{figure}[H]
			\caption{Line chart of portfolios returns}
			\begin{center}
				\includegraphics[width=75mm,scale=1]{port_ret_complete_small.png}
			\end{center}
		\end{figure}

	\section{Conclusions}
	As wee can see by the results the best performing portfolios are the ones built with Residuals variance and R-Squared as paramenters. Many portfolios have negative annual returns, probably this negative annual return is caused because the dataset contains daily data including the years 2020-2022; it is possible that theese bad results are caused by the covid 19 pandemic and the war in Ukraine.
\nocite{*} %aggiunge a bibliografia gli elementi non citati nel testo
\printbibliography[title={Bibliography}] %Prints bibliography
\end{document}
